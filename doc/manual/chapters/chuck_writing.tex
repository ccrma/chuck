\section{Tutorial 1: Writing To Disk}

Here is a brief tutorial on writing to disk...

---
1. recording your ChucK session to file is easy!

example:  you want to record the following:

\chuckterm{\prompt chuck foo.ck bar.ck}

all you's got to do is ChucK a shred that writes to file:

\chuckterm{\prompt chuck foo.ck bar.ck rec.ck}

no changes to existing files are necessary.
an example rec.ck can be found in examples/, this
guy/gal writes to ``foo.wav''.  edit the file to change.
if you don't want to worry about overwriting the same
file everytime, you can:

\chuckterm{\prompt chuck foo.ck bar.ck rec2.ck}

rec2.ck will generate a file name using the current
time.  You can change the prefix of the filename by
\begin{verbatim}
        "data/session" => w.autoPrefix;
\end{verbatim}
w is the WvOut in the patch.

Oh yeah, you can of course chuck the rec.ck on-the-fly...

from terminal 1
\chuckterm{\prompt chuck -\,-loop}

from terminal 2
\chuckterm{\prompt chuck + rec.ck}


---
2. silent mode

you can write directly to disk without having real-time audio
by using -\,-silent or -s

\chuckterm{\prompt chuck foo.ck bar.ck rec2.ck -s}

this will not synchronize to the audio card, and will generate
samples as fast as it can.


---
3. start and stop

you can start and stop the writing to file by:
\begin{verbatim}
        1 => w.record;  // start
        0 => w.record;  // stop
\end{verbatim}
as with all thing ChucKian, this can be done
sample-synchronously.


---
4. another halting problem

what if I have infinite time loop, and want to terminate
the VM, will my file be written out correctly?  the answer:

Ctrl-C works just fine.

ChucK STK module keeps track of open file handles and
closes them even upon abnormal termination, like Ctrl-C.
Actually for many, Ctrl-C is the natural way to end your
ChucK session.  At any rate, this is quite ghetto, but it works.
As for seg-faults and other catastrophic events, like computer
catching on fire from ChucK exploding, the file probably is
toast.

hmmmm, toast...


---
5. the silent sample sucker strikes again

as in rec.ck, one patch to write to file is:
\begin{verbatim}
        dac => Gain g => WvOut w => blackhole;
\end{verbatim}
the blackhole drives the WvOut, which in turns sucks
samples from Gain and then the dac.  The WvOut
can also be placed before the dac:
\begin{verbatim}
        Noise n => WvOut w => dac;
\end{verbatim}
The WvOut writes to file, and also pass through the incoming samples.
