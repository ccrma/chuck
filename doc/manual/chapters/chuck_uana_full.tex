\chapter{UAna objects}

\chuckugen{
\ugen{UAna} 
\begin{chuckitemize}
\item  Unit Analyzer base class
\end{chuckitemize}
\verbatiminput{examples/uana/uana_uana.txt}
extends UGen 

     \chuckmultifunction{UAnaBlob .upchuck() }{initiate analysis at the UAna returns result.}
}

\chuckugen{\ugen{[object]:UAnaBlob }
\begin{chuckitemize}
    \item Unit Analyzer blob for contain of data
\end{chuckitemize}
\verbatiminput{examples/uana/uana_uanablob.txt}

    \chuckmultifunction{ float .fval(int index) }{ get blob's float value at index}
    \chuckmultifunction{ complex .cval(int index) }{ get blob's complex value at index}
    \chuckmultifunction{ float[] .fvals() }{ get blob's float array}
    \chuckmultifunction{ complex[] .cvals() }{ get blob's complex array}
    \chuckmultifunction{ time .when() }{ get the time when blob was last upchucked}
}

\chuckugen{ 
\ugen{[object]: Windowing }
\begin{chuckitemize}
    \item Helper class for generating transform windows
\end{chuckitemize}
\verbatiminput{examples/uana/uana_window.txt}
    
    \chuckmultifunction{ float[] .rectangle(int length) }{ generate a rectangular window}
    \chuckmultifunction{ float[] .triangle(int length) }{ generate a triangular (or Barlett) window}
    \chuckmultifunction{ float[] .hann(int length) }{ generate a Hann window}
    \chuckmultifunction{ float[] .hamming(int length) }{ generate a Hamming window}
    \chuckmultifunction{ float[] .blackmanHarris(int length) }{ generate a blackmanHarris window}
examples: win.ck
}

\ugenheading{domain transformations}

\chuckugen{\ugen{[uana]: FFT} 
\begin{chuckitemize} 
    \item Fast Fourier Transform
\end{chuckitemize}
\verbatiminput{examples/uana/uana_fft.txt}
extends UAna 

    \chuckmultifunction{ .size  ( float, READ/WRITE ) }{ get/set the FFT size}
    \chuckmultifunction{ .window()  ( float[], READ/WRITE )  }{get/set the transform window/size
      (also see AAA Windowing)}
    \chuckmultifunction{ .windowSize  ( int, READ only ) }{ get the current window size}
    \chuckmultifunction{ .transform  ( float[], WRITE only ) }{ manually take FFT (as opposed to
      using .upchuck() / upchuck operator)}
    \chuckmultifunction{ .spectrum  ( complex[], READ only )  }{manually retrieve the results of a
      transform}

(UAna input/output)

    \chuckfunction{ {\bf input:} audio samples from an incoming UGen}
    \chuckfunction{ {\bf {\bf output}} spectrum in complex array, magnitude spectrum in float array}

examples: fft.ck, fft2.ck, fft3.ck win.ck
}

\chuckugen{\ugen{[uana]: IFFT} 
\begin{chuckitemize}
    \item Inverse Fast Fourier Transform
\end{chuckitemize}
\verbatiminput{examples/uana/uana_ifft.txt}
extends UAna 

    \chuckmultifunction{ .size - ( float, READ/WRITE ) }{ get/set the IFFT size}
    \chuckmultifunction{ .window() - ( float[], READ/WRITE ) }{ get/set the transform window/size (also see AAA Windowing)}
    \chuckmultifunction{ .windowSize - ( int, READ only ) }{ get the current window size}
    \chuckmultifunction{ .transform - ( complex[], WRITE only ) }{ manually take IFFT (as opposed to using .upchuck() / upchuck operator)}
    \chuckmultifunction{ .samples - ( float[], READ only ) }{ manually retrieve the result of the previous IFFT}

(UAna input/output)

    \chuckfunction{ {\bf input:} complex spectral frames (either via UAnae connected via \upchuckop, or manullay via .transform())}
    \chuckfunction{ {\bf output} audio samples (overlap-added and streamed out to UGens connected via \chuckop)}

examples: ifft.ck, fft2.ck, ifft3.ck
}

\chuckugen{\ugen{[uana]: DCT}
\begin{chuckitemize}
    \item  Discrete Cosine Transform
\end{chuckitemize}
\verbatiminput{examples/uana/uana_dct.txt}
extends UAna 

    \chuckmultifunction{ .size - ( float, READ/WRITE ) }{ get/set the DCT size}
    \chuckmultifunction{ .window() - ( float[], READ/WRITE ) }{ get/set the transform window/size (also see AAA Windowing)}
    \chuckmultifunction{ .windowSize - ( int, READ only )  }{get the current window size}
    \chuckmultifunction{ .transform - ( float[], WRITE ) }{ manually take DCT (as opposed to using .upchuck() / upchuck operator)}
    \chuckmultifunction{ .spectrum - ( float[], READ only ) }{ manually retrieve the results of a transform}

(UAna input/output)

    \chuckfunction{ {\bf input:} audio samples (either via UAnae connected via \upchuckop, or manullay via .transform())}
    \chuckfunction{ {\bf output} discrete cosine spectrum}

examples: dct.ck
}

\chuckugen{ \ugen{[uana]: IDCT }
\begin{chuckitemize}
    \item  Inverse Discrete Cosine Transform
\end{chuckitemize}
\verbatiminput{examples/uana/uana_idct.txt}
extends UAna 

    \chuckmultifunction{ .size - ( float, READ/WRITE ) }{ get/set the IDCT size}
    \chuckmultifunction{ .window() - ( float[], READ/WRITE ) }{ get/set the transform window/size (also see AAA Windowing)}
    \chuckmultifunction{ .windowSize - ( int, READ only ) }{ get the current window size}
    \chuckmultifunction{ .transform - ( float[], WRITE ) }{ manually take IDCT (as opposed to using .upchuck() / upchuck operator)}
     \chuckmultifunction{ .samples - ( float[], WRITE ) }{ manually get result of previous IDCT}

(UAna input/output)

    \chuckfunction{ {\bf input:} real-valued spectral frames (either via UAnae connected via \upchuckop, or manullay via .transform())}
    \chuckfunction{ {\bf output} audio samples (overlap-added and streamed out to UGens connected via \chuckop)}

examples: idct.ck
}

\ugenheading{feature extractors}

\chuckugen{\ugen{[uana]: Centroid} 
\begin{chuckitemize}
    \item  Spectral Centroid
\end{chuckitemize}
\verbatiminput{examples/uana/uana_spectral_centroid.txt}
extends UAna 

    \chuckmultifunction{float .compute(float[]) }{ manually computes the centroid from a float array}

(UAna input/output)

    \chuckfunction{ {\bf input:} complex spectral frames (e.g., via UAnae connected via \upchuckop)}
    \chuckfunction{ {\bf output} the computed Centroid value is stored in the blob's floating
      point vector, accessible via .fval(0). This is a normalized value in the
      range (0,1), mapped to the frequency range 0Hz to Nyquist}

examples: centroid.ck
}

\chuckugen{\ugen{[uana]: Flux }
\begin{chuckitemize}
    \item Spectral Flux
\end{chuckitemize}
\verbatiminput{examples/uana/uana_spectral_flux.txt}
extends UAna 

    \chuckmultifunction{void .reset( ) }{ reset the extractor}
    \chuckmultifunction{float .compute(float[] f1, float[] f2) }{ manually computes the flux
      between two frames}
    \chuckmultifunction{float .compute(float[] f1, float[] f2, float[] diff)  }{manually computes
      the flux between two frames, and stores the difference in a third array}

(UAna input/output)

    \chuckfunction{ {\bf input:} complex spectral frames (e.g., via UAnae connected via \upchuckop)}
    \chuckfunction{ {\bf output} the computed Flux value is stored in the blob's floating point
      vector, accessible via .fval(0)}

examples: flux.ck, flux0.ck
}

\chuckugen{\ugen{[uana]: RMS }
\begin{chuckitemize}
    \item  Spectral RMS
\end{chuckitemize}
\verbatiminput{examples/uana/uana_spectral_rms.txt}
extends UAna 

    \chuckmultifunction{float .compute(float[]) }{ manually computes the RMS from a float array}

(UAna input/output)

    \chuckfunction{ {\bf input:} complex spectral frames (e.g., via UAnae connected via \upchuckop)}
    \chuckfunction{ {\bf output} the computed RMS value is stored in the blob's floating point vector, accessible via .fval(0)}

examples: rms.ck
}

\chuckugen{\ugen{[uana]: RollOff }
\begin{chuckitemize}
    \item  Spectral RollOff
\end{chuckitemize}
\verbatiminput{examples/uana/uana_spectral_rolloff.txt}
extends UAna 

    \chuckmultifunction{ float .percent((float val)) }{ set the percentage for computing rolloff}
    \chuckmultifunction{ float .percent(( )) }{ get the percentage specified for the rolloff}
    \chuckmultifunction{ float .compute(float[]) }{ manually computes the rolloff from a float array}

(UAna input/output)

    \chuckfunction{ {\bf input:} complex spectral frames (e.g., via UAnae connected via \upchuckop)}
    \chuckfunction{ {\bf output} the computed rolloff value is stored in the blob's floating point
      vector, accessible via .fval(0). This is a normalized value in the range
      [0,1), mapped to the frequency range 0 to nyquist frequency.}
examples: rolloff.ck
}
