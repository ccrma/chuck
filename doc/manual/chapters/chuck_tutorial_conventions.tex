\section{Conventions}

ChucK is supported under many different operating systems. While ChucK code is intended to be truly "platform-independent", each different OS has their own ``features'' that make the experience of working with ChucK slightly different. This chapter will outline some of these differences. 

%(no, ChucK is not sick, but it may cause nausea, stress, frustration...). 
ChucK is used as a terminal application in this tutorial, so you will need to know how to access and navigate in the terminal. Here are some hints about getting started with the terminal on your operating system.


\subsection{OS X}

The terminal is located in the Utilities/ folder in the Applications/ folder of your hard drive. Double click on Terminal. You can click and hold on the icon in the Dock and select the ``Keep in Dock'' option. Now the Terminal application will be conveniently located in the Dock. 

\href{http://www.macdevcenter.com/pub/ct/51}{http://www.macdevcenter.com/pub/ct/51}

\href{http://www.atomiclearning.com/macosxterminalx.shtml}{http://www.atomiclearning.com/macosxterminalx.shtml}

\subsection{Windows}

The terminal is accessed by clicking on the Start Menu and then clicking on run. In the window that opens type {\bf cmd}. 

\href{http://www.c3scripts.com/tutorials/msdos/}{http://www.c3scripts.com/tutorials/msdos/}

\href{http://www.ss64.com/nt/}{http://www.ss64.com/nt/}

\subsection{Linux}

No hints needed here.
% If you are reading this then you are required to return your tinfoil hat.
