\chapter{Control Structures}

ChucK includes standard control structures similar to those in most programming languages. A condition (of type 'int') is evaluated and then a proceeding block is potentially executed. Blocks are separated either by semicolons or by curly brackets {}.

 

\section{if / else}

The if statement executes a block if the condition is evaluated as non-zero.
\begin{verbatim}
    if( condition )
    {
        // insert code here
    }
\end{verbatim}
 In the above code, condition is any expression that evaluates to an int.

The else statement can be put after the if block to handle the case where the condition evaluates to 0.
\begin{verbatim}
    if( condition )
    {
        // your code here
    }
    else
    {
        // your other code here
    }
\end{verbatim}

If statements can be nested.
 

\section{while}

The while statement is a loop that repeatedly executes the body as long as the condition evaluates as non-zero.
\begin{verbatim}
    // here is an infinite loop
    while( true )
    {
        // your code loops forever!

        // (sometimes this is desirable because we can create
        // infinite time loops this way, and because we have
        // concurrency)
    } 
\end{verbatim}

The while loop will first check the condition, and executes the body as long as the condition evaluates as non-zero. To execute the body of the loop before checking the condition, you can use a do/while loop. This guarantees that the body gets executed as least once.
\begin{verbatim}
    do {
        // your code executes here at least once
    } while( condition );
\end{verbatim}
 A few more points:
\begin{chuckitemize}
\item while statements can be nested.
\item see break/continue for additional control over your loops
\end{chuckitemize}

\section{until}

The until statement is the opposite of while, semantically. A until loop repeatedly executes the body until the condition evaluates as non-zero.
\begin{verbatim}
    // an infinite loop
    until( false )
    {
        // your great code loops forever!
    }
\end{verbatim}

The while loop will first check the condition, and executes the body as long as the condition evaluates as zero. To execute the body of the loop before checking the condition, you can use a do/until loop. This guarantees that the body gets executed as least once.
\begin{verbatim}
    do {
        // your code executes here at least once
    } until( condition );
\end{verbatim}

 A few more points:
\begin{chuckitemize}
\item until statements can be nested.
\item see break/continue for additional control over your loops
\end{chuckitemize}

\section{for}

A loop that iterates a given number of times. A temporary variable is declared that keeps track of the current index and is evaluated and incremented at each iteration.
\begin{verbatim}
    // for loop
    for( 0 => int foo; foo < 4 ; foo++ )
    {
        // debug-print value of 'foo'
        <<<foo>>>;
    }
\end{verbatim}
 
\section{break / continue}

Break allows the program flow to jump out of a loop.
\begin{verbatim}
    // infinite loop
    while( 1 )
    {
        if( condition ) 
            break;
    }
\end{verbatim}

 Continue allows a loop to continue looping but not to execute the rest of the block for the iteration where continue was executed.
\begin{verbatim}
    // another infinite loop
    while( 1 )
    {
        // check condition
        if( condition )
            continue;

        // some great code that may get skipped (if continue is taken)
}
\end{verbatim}