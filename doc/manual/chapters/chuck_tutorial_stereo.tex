\section{Stereo}
\textbf{by Adam Tindale}\\

At long last, ChucK is stereo! Accessing the stereo capabilities of ChucK is relatively simple. {\bf dac} now has three access points. 

\begin{verbatim}
    UGen u;
    // standard mono connection
    u => dac;

    // simple enough!
    u => dac.left;
    u => dac.right;
\end{verbatim}

{\bf adc} functionality mirrors dac. 

\begin{verbatim}
    // switcheroo
    adc.right => dac.left;
    adc.left => dac.right;
\end{verbatim}

If you have your great UGen network going and you want to throw it somewhere in the stereo field you can use {\bf Pan2}. You can use the {\bf .pan} function to move your sound between left (-1) and right (1).

\begin{verbatim}
    // this is  a stereo connection to the dac
    SinOsc s => Pan2 p => dac;
    1 => p.pan;
    while(1::second => now){
        // this will flip the pan from left to right
        p.pan() * -1. => p.pan;
    }
\end{verbatim}


You can also mix down your stereo signal to a mono signal using the {\bf Mix2} object. 

\begin{verbatim}
   adc => Mix2 m => dac.left;
\end{verbatim}

If you remove the Mix2 in the chain and replace it with a Gain object it will act the same way. When you connect a stereo object to a mono object it will sum the inputs. You will get the same effect as if you connect two mono signals to the input of another mono signal. 