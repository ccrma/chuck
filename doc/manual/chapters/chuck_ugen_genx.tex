\chuckugen{
\ugen{GenX}
\begin{chuckitemize}
    \item base class for classic MusicN lookup table unit generators
    \item see examples: readme-GenX.ck (examples/special/readme-GenX.ck)
\end{chuckitemize}

\verbatiminput{examples/ugen/GenX.txt}


\control
\begin{chuckitemize}
    \item {\bf .lookup} - ( float , READ ONLY ) - {\it returns lookup table value at index [ -1, 1 ]; absolute value is used in the range [ -1, 0 ]}
    \item {\bf .coefs} - ( float [ ] , WRITE ONLY ) - {\it set lookup table coefficients; meaning is dependent on subclass}
\end{chuckitemize}
}

\chuckugen{
\ugen{Gen5}
\begin{chuckitemize}
    \item exponential line segment lookup table table generator
    \item see examples: Gen5-test.ck (examples/special/Gen5-test.ck)
\end{chuckitemize}
\verbatiminput{examples/ugen/Gen5.txt}


extends GenX

\control

   see GenX
}

\chuckugen{
\ugen{Gen7}
\begin{chuckitemize}
    \item line segment lookup table table generator
    \item see examples: Gen7-test.ck (examples/special/Gen7-test.ck)
\end{chuckitemize}

\verbatiminput{examples/ugen/Gen7.txt}


extends GenX


\control

   see GenX
}

\chuckugen{
\ugen{Gen9}
      
\begin{chuckitemize}
    \item sinusoidal lookup table with harmonic ratio, amplitude, and phase control
    \item see examples: Gen9-test.ck (examples/special/Gen9-test.ck)
\end{chuckitemize}


\verbatiminput{examples/ugen/Gen9.txt}


       extends GenX

\control

   see GenX
}

\chuckugen{
\ugen{Gen10}
\begin{chuckitemize}
    \item  sinusoidal lookup table with partial amplitude control
    \item  see examples: Gen10-test.ck (examples/special/Gen10-test.ck)
\end{chuckitemize}

\verbatiminput{examples/ugen/Gen10.txt}



extends GenX

\control

   see GenX
}

\chuckugen{
\ugen{Gen17}
\begin{chuckitemize}
    \item chebyshev polynomial lookup table
    \item see examples: Gen17-test.ck (examples/special/Gen17-test.ck)
\end{chuckitemize}

\verbatiminput{examples/ugen/Gen17.txt}



extends GenX

\control

   see GenX
}

\chuckugen{
\ugen{CurveTable}
\begin{chuckitemize}
    \item flexible curve/line segment table generator
    \item see examples: GenX-CurveTable-test.ck
      (examples/special/GenX-CurveTable-test.ck)
\end{chuckitemize}

\verbatiminput{examples/ugen/Curvetable.txt}



extends GenX

\control

see GenX

}

\chuckugen{
\ugen{LiSa}
\begin{chuckitemize}
    \item live sampling utility.
\end{chuckitemize}

\verbatiminput{examples/ugen/LiSa.txt}

\control
\begin{chuckitemize}

    \item {\bf .duration} - ( dur , READ/WRITE ) - sets buffer size; required to
      allocate memory, also resets all parameter values to default
    \item {\bf .record} - ( int , READ/WRITE ) - turns recording on and off
    \item {\bf .getVoice} - ( READ ) - returns the voice number of the next
      available voice
    \item {\bf .maxVoices} - ( int , READ/WRITE ) - sets the maximum number of
      voices allowable; 10 by default (200 is the current hardwired
      internal limit)
    \item {\bf .play} - ( int, WRITE ) - turn on/off sample playback /(voice 0) /
    \item {\bf .play} - ( int voice, int, WRITE) - for particular voice (arg 1),
      turn on/off sample playback
    \item {\bf .rampUp} - ( dur, WRITE ) - turn on sample playback, with ramp
      /(voice 0) /
    \item {\bf .rampUp} - ( int voice dur, WRITE ) - for particular voice (arg
      1), turn on sample playback, with ramp
    \item {\bf .rampDown} - ( dur, WRITE ) - turn off sample playback, with ramp
      /(voice 0) /
    \item {\bf .rampDown} - ( int voice, dur, WRITE ) - for particular voice
      (arg 1), turn off sample playback, with ramp
    \item {\bf .rate} - ( float, WRITE ) - /set playback rate (voice 0). Note
      that the int/float type for this method will determine whether the
      rate is being set (float, for voice 0) or read (int, for voice
      number)/
    \item {\bf .rate} - ( int voice, float, WRITE ) - for particular voice (arg
      1),/ set playback rate/
    \item {\bf .rate} - ( READ ) - /get playback rate (voice 0) /
    \item {\bf .rate} - ( int voice, READ ) - for particular voice (arg 1), /get
      playback rate. Note that the int/float type for this method will
      determine whether the rate is being set (float, for voice 0) or
      read (int, for voice number) /
    \item {\bf .playPos} - ( dur, WRITE ) - /set playback position (voice 0) /
    \item {\bf .playPos} - ( int voice, dur, WRITE ) - for particular voice (arg
      1), /set playback position/
    \item {\bf .recPos} - ( dur, READ/WRITE ) - /get/set record position /
    \item {\bf .recRamp} - ( dur , READ/WRITE ) - set ramping when recording
      (from 0 to loopEndRec)
    \item {\bf .loopRec} - ( int, READ/WRITE ) - /turn on/off loop recording /
    \item {\bf .loopEndRec} - ( dur, READ/WRITE ) - /set end point in buffer for
      loop/ /recording /
    \item {\bf .loopStart} - ( dur , READ/WRITE ) - set loop starting point for
      playback (voice 0). only applicable when 1 => loop.
    \item {\bf .loopStart} - ( int voice, dur , WRITE ) - for particular voice
      (arg 1), set loop starting point for playback. only applicable
      when .loop(voice, 1).
    \item {\bf .loopEnd} - ( dur , READ/WRITE ) - set loop ending point for
      playback (voice 0). only applicable when 1 => loop.
    \item {\bf .loopEnd} - ( int voice, dur , WRITE ) - for particular voice
      (arg 1), set loop ending point for playback. only applicable when
      .loop(voice, 1).
    \item {\bf .loop} - ( int , READ/WRITE ) - turn on/off looping (voice 0)
    \item {\bf .loop} - ( int voice, int, READ/WRITE ) - for particular voice
      (arg 1), turn on/off looping
    \item {\bf .bi} - ( int , READ/WRITE ) - turn on/off bidirectional playback
      (voice 0)
    \item {\bf .bi} - ( int voice, int , WRITE ) - for particular voice (arg 1),
      turn on/off bidirectional playback
    \item {\bf .voiceGain} - ( float , READ/WRITE ) - set playback gain (voice 0)
    \item {\bf .voiceGain} - ( int voice, float , WRITE ) - for particular voice
      (arg 1), set gain
    \item {\bf .feedback} - ( float , READ/WRITE ) - get/set feedback amount
      when overdubbing (loop recording; how much to retain)
    \item {\bf .valueAt} - ( dur, READ ) - /get value directly from record buffer /
    \item {\bf .valueAt} - ( sample, dur, WRITE ) - /set value directly in
      record buffer /
    \item {\bf .sync} - (int, READ/WRITE) - /set input mode; (0) input is
      recorded to internal buffer, (1) input sets playback position
      [0,1] (phase value between loopStart and loopEnd for all active
      voices), (2) input sets playback position, interpreted as a time
      value in samples (only works with voice 0) /
    \item {\bf .track} - (int, READ/WRITE) - /identical to sync /
    \item {\bf .clear} - /clear recording buffer /
\end{chuckitemize}

}
